%Definitions regarding the document 
\documentclass[11pt, a4paper]{article}
\usepackage{geometry}
 \geometry{
 a4paper,
 %total={170mm,257mm},
 left=15mm,
 top=20mm,
 right=15mm,
 bottom=20mm,
 }
\title{Questions for Assignment}
\author{Shahadat Hussain Parvez}
\date{November 2020}
\newcommand{\examName}{Assignment for Mid Semester Examination Fall 2020} %Assignment Title
\newcommand{\courseCode}{CSE 000} %Course Code
\newcommand{\courseTitle}{Test Course} %Course Title
\newcommand{\totalMarks}{15} %Marks 
\newcommand{\totalTime}{24 Hours} %Time Allowed for 
%Add all instructions here
\newcommand{\instructions}{Answer all the questions.\\
Marks for this test will be provided holistically instead of for parts of the questions. Also marks for this test will vary depending on the viva performance.\\
%add more instructions as necessary
Try to answer parts of a question together.
}

%Definitions regarding Assignment submission. These are used in the submissionInstructions Page
\newcommand{\emailAddress}{shparvez@neub.edu.bd} %Email address for assignment submission
\newcommand{\emailSubject}{CSE 121 Midterm Assignment Fall 2020 by 2001030200XX } %Prescribed email subject for submission
\newcommand{\fileName}{CSE121Mid-2001030200XX.pdf } %Prescribed filename for submission
\newcommand{\submissionDeadline}{ December 2, 2020, Wednesday, 02:00 PM } %Last date of Submission
\newcommand{\noticeDeadline}{ November 30, 2020, Wednesday} %Deadline to notify you about medical condition



\usepackage{amsmath,amssymb}
\usepackage{multicol}
%\usepackage{filecontents}
\usepackage{hyperref}
\usepackage{lipsum}
\usepackage{blindtext}

\usepackage{catchfile,environ,tikz}
\usepackage{verbatim}

%Begining of randomizer
%%https://tex.stackexchange.com/questions/174872/generate-exam-from-a-question-bank
\makeatletter% Taken from https://tex.stackexchange.com/q/109619/5764
\def\declarenumlist#1#2#3{%
  \expandafter\edef\csname pgfmath@randomlist@#1\endcsname{#3}%
  \count@\@ne
  \loop
    \expandafter\edef
    \csname pgfmath@randomlist@#1@\the\count@\endcsname
      {\the\count@}
    \ifnum\count@<#3\relax
    \advance\count@\@ne
  \repeat}
\def\prunelist#1{%
  \expandafter\xdef\csname pgfmath@randomlist@#1\endcsname
          {\the\numexpr\csname pgfmath@randomlist@#1\endcsname-1\relax}
  \count@\pgfmath@randomtemp 
  \loop
    \expandafter\global\expandafter\let
    \csname pgfmath@randomlist@#1@\the\count@\expandafter\endcsname
    \csname pgfmath@randomlist@#1@\the\numexpr\count@+1\relax\endcsname
    \ifnum\count@<\csname pgfmath@randomlist@#1\endcsname\relax
      \advance\count@\@ne
  \repeat}
\makeatother

% Define how each questionblock should be handled
\newcounter{questionblock}
\newcounter{totalquestions}
\NewEnviron{questionblock}{}%

\newcommand{\randomquestionsfrombank}[2]{%
  \CatchFileDef{\bank}{#1}{}% Read the entire bank of questions into \bank
  \setcounter{totalquestions}{0}% Reset total questions counters  ***
  \RenewEnviron{questionblock}{\stepcounter{totalquestions}}% Count every question  ***
  \bank% Process file  ***
  \declarenumlist{uniquequestionlist}{1}{\thetotalquestions}% list from 1 to totalquestions inclusive.
  \setcounter{totalquestions}{#2}% Start the count-down
  \RenewEnviron{questionblock}{%
    \stepcounter{questionblock}% Next question
    \ifnum\value{questionblock}=\randomquestion 
      \par% Start new paragraph
      \item \BODY% Print question
    \fi
  }%
  \foreach \uNiQueQ in {1,...,#2} {% Extract #2 random questions
    \setcounter{questionblock}{0}% Start fresh with question block counter
    \pgfmathrandomitem\randomquestion{uniquequestionlist}% Grab random question from list
    \xdef\randomquestion{\randomquestion}% Make random question available globally
    \prunelist{uniquequestionlist}% Remove picked item from list
    \bank% Process file
  }}
%End of randomizer






%Actual Document starts
\begin{document}

%Define the number of sets here. Change 10 to any number you need
\newcommand{\numberOfSets}{10}
\foreach \x in {1,...,\numberOfSets} {
\setcounter{page}{1}    %Define page number 1 for each sets

%Set header starts here
    \large Set: \x
    \begin{center}
        \Large North East University Bangladesh
        \\Department of Computer Science and Engineering
        \\ \examName
        \\Program: B.Sc. (Eng.) in CSE
        \\Course: \courseCode(\courseTitle)
    \end{center}
    \begin{tabular}{l c r} 
        {\Large Marks:  \totalMarks}& 
        {\hspace{9cm} }& 
        {\Large Total Time:   \totalTime}\\
        \hline
    \end{tabular}\\
    \textbf{Instructions: \instructions}
%Set header ends here

%Start of questions    
    \bigskip
    \begin{center}
       Answer the following questions.
    \end{center} 
    
    %Start of question numbering    
    \begin{enumerate}
    %Simple example of using dumping questions randomly
    \randomquestionsfrombank{1. firstType.tex}{5}
    
    %Fixed questions can be also added inside an item tage.
    \item This is a demo question
    
    %Demo of generating second level Questions.  For this inside an item enumeration needed.
    \item In answer script write the statement first and identify if the statements are true or false. 
    \begin{enumerate}
        \randomquestionsfrombank{2. secondType.tex}{5}
    \end{enumerate}   
    
    
    %Adding different instructions for different type of questions
    \bigskip
    \begin{center}
        Answer each parts of the following questions
    \end{center}
    \randomquestionsfrombank{3. thirdType.tex}{2}
    \end{enumerate}
    %End of questions
    \begin{center}
        \Huge{---End of Questions---}
    \end{center}
    
    %Add the Submission Instruction Page    
    \pagebreak
    %SUbmission Instruction
\begin{center}
    \Huge{Submission Instruction}
\end{center}
\bigskip

\large{
The last date to submit this assignment is \submissionDeadline. No late submissions will be allowed. All the answers must be handwritten and scanned and sent as PDF files. No other file formats will be allowed.\\

If you have any health condition refraining you from submitting on time, you have to inform me before \noticeDeadline for prior approval for late submission. It will to my discretion to allow you for late submission based on the severity of your condition.\\

You should only submit the questions from set you are assigned. Submission of any other set as well as copying from other students answers will be heavily penalized.\\

Use the attached format for front Page. The front page can be printed or handwritten.\\

The procedure to submit assignments is as follows:
\begin{itemize}
    \item Prepare your assignment by scanning the handwritten assignments and converting them to PDF. The pages should be in order. No effort will be made to check disordered files or late submissions. All the pages should be in a single PDF file.
    \item Rename your file as \textbf{\fileName}, where the last 12 digits are your registration number. Any other format of naming will not be allowed and will not be considered for grading.
    \item Send your assignment to the following address \href{mailto:shparvez@neub.edu.bd?subject=\emailSubject }{shparvez@neub.edu.bd} . There is no need to CC or send any copy to my other email address.
    \begin{itemize}
        \item The subject line of your email should read as \textbf{\emailSubject} . Where the last 12 digits are your registration number. Any other format will eliminate the chance of your email to reach the correct folder of my email. You are suggested to copy the text if necessary.
    \end{itemize}
\end{itemize}

}

    %Add the Demo front Page
    \pagebreak
    %Front Page demo
\vspace{5cm}    

\begin{center}
    \LARGE North East University Bangladesh
    \\Department of Computer Science and Engineering
    \\ \examName
    \\Program: B.Sc. (Eng.) in CSE
    \\Course: \courseCode(\courseTitle)
\end{center}

\vspace{5cm}    
\LARGE{
Name: [[Your Name goes Here]]

Reg: [[Your registration number goes Here]]
}

    
}

\end{document}